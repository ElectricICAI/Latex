\documentclass[a4paper,titlepage]{article}
\usepackage[utf8]{inputenc}
\usepackage[spanish]{babel}
\title{}
\author{}
\date{}
\pagestyle{empty}
\usepackage{natbib}
\usepackage{graphicx}
\usepackage{tikz,pgfplots}
\usepackage{subcaption}
\usepackage[hidelinks]{hyperref}
\usepackage[a4paper,margin=3cm]{geometry}
\usepackage{tikz-3dplot}
\spanishdecimal{.}
\usepackage[hidelinks]{hyperref}
\usepackage{fancyhdr}

\usepackage{multirow}

%%%%%%%% Matematicas
\usepackage{amsmath}
\usepackage{amsfonts}
\usepackage{amssymb}
\newcommand{\p}{ \cdot }
\addto\captionsspanish{
			\def\contentsname{Índice general}
			\def\listtablename{Índice de tablas}
			\def\tablename{Tabla}
		}
\title{Ejes de coordenadas y gráficas}
		
\begin{document}

\maketitle

\section{Ejes cartesianos}
\begin{center}
    \begin{tikzpicture}[scale=1.25]
    \begin{axis}[
      axis lines=middle,
      axis line style={very thick},
      xmin=-0.5,xmax=5.5,ymin=-0.5,ymax=5.5,
      xtick distance=1,
      ytick distance=1,
      xlabel=$x$ (\textit{km}),
      ylabel=$y$ (\textit{km}),
      y label style={at={(0.1,1)}},
      x label style={at={(1.1,0.1)}},
      title={\textit{Distribución del sistema a estudiar}},
      grid=major,
      grid style={densely dashed,black!20}];
      \addplot[mark=empty,fill=green] coordinates {(2.5,2.5)}
      node[label=above:{\small Fábrica $(2.5,2.5)$},circle, fill, inner sep=2pt,name=A] {}; 
      \addplot[mark=empty,fill=cyan] coordinates {(1,2.5)}
      node[label=below:{\small Térmica $(1,2.5)$},circle, fill, inner sep=2pt,name=A] {}; 
      \addplot[mark=empty,fill=red] coordinates {(4,0.5)}
      node[label=above:{\small Población $(4,0.5)$},circle, fill, inner sep=2pt,name=A] {};
   \end{axis}
    \end{tikzpicture}
\end{center}

\section{Ejes en polares}
\begin{center}
\begin{tikzpicture}[=latex]

    % Draw the lines at multiples of pi/12
    \foreach \ang in {0,...,31} {
      \draw [lightgray, very thin] (0,0) -- (\ang * 180 / 16:4);
    }
    
    % Concentric circles and radius labels
    \foreach \s in {0, 1, 2, 3} {
      \draw [lightgray, very thin] (0,0) circle (\s + 0.5);
      \draw[lightgray, very thin] (0,0) circle (\s);
    }
    \draw[blue] (0,2.5) arc (90:75:2.5);
    \draw[red] (0,4) arc (90:-250:4);
    % Add the labels at multiples of pi/4
    \foreach \ang/\lab/\dir in {
      0/{\small\textit{Este}}/{right},
      1/{\small\textit{Noreste}}/{above right},
      2/{\small\textit{Norte}}/above,
      3/{\small\textit{Noroeste}}/{above left},
      4/{\small\textit{Oeste}}/left,
      5/{\small\textit{Suroeste}}/{below left},
      7/{\small\textit{Sureste}}/{below right},
      6/{\small\textit{Sur}}/below} {
      \draw (0,0) -- (\ang * 180 / 4:4.1);
      \node [fill=white] at (\ang * 180 / 4:4.2) [\dir] {\scriptsize $\lab$};
      
    }
    
    % The double-lined circle around the whole diagram
    \draw [style=double, very thin] (0,0) circle (4.1);
    \draw[red] (fov) -- ++(110:4cm) circle (2pt) node[above left] {\textit{B} (4 km, 340º)};
     \draw[blue] (fov) -- ++(75:2.5cm) circle (2pt) node[above right] {\textit{A} (2.5 km, 15º)};
     \draw[black] (fov) -- ++(0:0cm) circle (2pt) node[below] {\textit{Punto de estudio} (0 km, 0º)};
    %\node[circle,draw] (u\a) at ({\a*30}:2){};
    %\node [vertex]  at ( 15 * 180 / 2.5:3.5) [\dir] {\textcolor{red}{\textit{A} (2.5 km, 15º)}};
    %\node [fill=white,vertex] at ( / 2.5:3.5) [\dir] {\textcolor{red}{\textit{A} (4 km, 340º)}};
    \end{tikzpicture} 
\end{center}
\newpage
\section{Graficas con valores simbólicos}
\begin{center}
    \begin{tikzpicture}{h}
        \pgfplotsset{tick style={very thin,white}}
        \begin{axis}
            [width = \textwidth, axis lines = center,height=10cm, xmin=0, xmax=5, ymin=-1.25, ymax=1.25, xlabel ={$x\: (m)$}, ylabel=$V_z(x)$,xmajorgrids=true,xtick={0,1,2,3,4,4.5},xticklabels={0,0.1,0.2,0.3,0.4,0.45},ytick={-0.5,0,0.4}, yticklabels={$-\frac{5}{24}P\cos(\theta)$,0 ,$\frac{4}{24}P\cos(\theta)$ },
            ytick pos=left,yticklabel pos=left]
            \addplot[domain=0:2,cyan,very thick](\x,-0.5);\addlegendentry{$V_z(x)$};
            \addplot[domain=2:4.5,cyan,very thick](\x,0.4);
        \end{axis}
    \end{tikzpicture}
\end{center}

\begin{center}
    \begin{tikzpicture}{h}
        \pgfplotsset{tick style={very thin,white}}
        \begin{axis}
            [width = 13.5cm,height=9cm, axis lines = center, xmin=0, xmax=3.5, ymin=-1.25, ymax=1.25, xlabel ={$x\: (m)$}, ylabel=$N(x)$,xmajorgrids=true,xtick={0,1,2,3},xticklabels={0,4,8,12},ytick={-0.3,0,-0.8}, yticklabels={$-\alpha\Delta T AE+\frac{F_2}{3}$,0 ,$-\alpha\Delta T AE-\frac{2F_2}{3}$ },
            ytick pos=left,yticklabel pos=left]
            \addplot[domain=0:2,cyan,very thick](\x,-0.3);\addlegendentry{$N(x)$};
            \addplot[domain=2:3,cyan,very thick](\x,-0.8);
        \end{axis}
    \end{tikzpicture}
\end{center}

\section{Gráficas en 3D}
\begin{center}
    \tdplotsetmaincoords{70}{130}
    \begin{tikzpicture}[tdplot_main_coords, scale=1.5]
    \draw[<-](5,0,0)node[below left]{$x$}--(-0.5,0,0);
    \draw[<-](0,5,0)node[below left]{$y$}--(0,-0.5,0);
    \draw[<-](0,0,5)node[below left]{$z$}--(0,0,-0.5);
    \draw[-](3,0,0)--(3,3,0);
    \draw[-](0,3,0)--(3,3,0);
    \draw[-](3,3,0)--(3,3,3);
    \draw[-](0,3,3)--(3,3,3);
    \draw[-](3,0,3)--(3,3,3);
     \draw[-](3,0,3)--(3,0,0);
     \draw[-](0,3,3)--(0,3,0);
     \draw[-](0,3,3)--(0,0,3);
     \draw[-](3,0,3)--(0,0,3);
    \draw[<-](6,1.5,1.5)node[above left]{$\sigma_{x}$}--(3,1.5,1.5);
     \draw[<-](3,2.8,2.8)node[left]{$\tau_{\theta,x}$}--(3,1.5,1.5);
     \coordinate (A) at (3,1.5,1.5);
     \draw[fill=black] (A) circle (1pt);
     
    \end{tikzpicture}
\end{center}



\end{document}
